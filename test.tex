\documentclass[paper=b5j]{jlreq}
\usepackage{listings, jvlisting, bxpapersize, amssymb, amsmath, float, tablefootnote, amsfonts, refcount, jlisting}
\usepackage[dvipdfmx]{graphicx}


\lstset{
  basicstyle={\ttfamily},
  identifierstyle={\small},
  commentstyle={\smallitshape},
  keywordstyle={\small\bfseries},
  ndkeywordstyle={\small},
  stringstyle={\small\ttfamily},
  frame={tb},
  breaklines=true,
  columns=[l]{fullflexible},
  numbers=left,
  xrightmargin=0zw,
  xleftmargin=3zw,
  numberstyle={\scriptsize},
  stepnumber=1,
  numbersep=1zw,
  lineskip=-0.5ex
}

\renewcommand{\lstlistingname}{Source}

\begin{document}

\title{\TeX 入門}
\author{Standard}
\maketitle

\section{What is \TeX ?}

ようこそみなさん \TeX の世界へ。\TeX \footnote{"テックス"ではなく"テフ"や"テック"と読む} とは複雑な数式を含む文章を簡単に作成するためのソフトです。
ただ、Wordのように決まったエディタがあるわけでもなく、書いたものがそのままの形で見えるわけでもなく、一種のプログラミング言語のようにソースコードを記述して、それを \TeX コンパイラを用いてpdfなどの閲覧用のファイル形式に変換して作成します\footnote{ここで、\LaTeX という別のソフトウェアがあります。これは \TeX を用いてもっと簡単に論文やレポートを作成するためのものです。通常 \TeX と言ったらこっちの \LaTeX を指します。}。
なぜWordではなく \TeX を使うのかというと、Wordなどにも数式を記述する機能は備わっていますが、その機能がレポートや論文を書くにはあまりにも貧弱すぎ、また、フォントも見にくいものになってしまいます。そこでこの\TeX を使うのです。

\TeX を使えば例えばこんな数式も

\[
  ax^2+bx+c=0 \; (a\neq0)
\]
\[
  x=\frac{-b\pm\sqrt{b^2-4ac}}{2a}
\]

こんな数式も

\[
  \lim_{n \to \infty} \frac{1}{n} \sum_{k = 1}^{n} f \left( \frac{k}{n} \right) =
  \int_{0}^{1} f(x) dx
\]

こーんな数式だって簡単に書けます。

\[
  L = \int_{\alpha}^{\beta} \sqrt{\left( \frac{dx}{dt} \right) ^ 2 + \left( \frac{dy}{dt} \right) ^ 2}
\]

これらの数式はこのように記述することで作成できます。

\begin{lstlisting}[caption=LaTeX]
\[
  ax^2+bx+c=0 \; (a\neq0)
\]
\[
  x=\frac{-b\pm\sqrt{b^2-4ac}}{2a}
\]

\[
  \lim_{n \to \infty} \frac{1}{n} \sum_{k = 1}^{n} f \left( \frac{k}{n} \right) =
  \int_{0}^{1} f(x) dx
\]

\[
  L = \int_{\alpha}^{\beta} \sqrt{\left( \frac{dx}{dt} \right) ^ 2 + \left( \frac{dy}{dt} \right) ^ 2}
\]
\end{lstlisting}

\section{How to install?}

それでは、\TeX のインストール方法を説明します。Debian系Linux向けの説明しかしないので、Windows使いの方はWSL2をインストールして下さい\footnote{Windows環境に直接入れる方法もありますが、不具合やいろいろなパッケージを入れるときに不便なので、WSL上でのインストールをおすすめします。もし、それでもWindowsにインストールしたかったらTeX Live Windows等で検索してみて下さい。}。

aptで以下のパッケージをインストールするだけで\TeX 自体のインストールは完了します\footnote{Ubuntu 24.04で確認済み}。

\begin{center}
  \$\ sudo apt install -y texlive-lang-japanese  texlive-latex-extra xdvik-ja \\ latexmk texlive-extra-utils
\end{center}

\section{How to use?}

\TeX ファイルは".tex"というファイルにテキスト形式で保存されます。
今回は例としてtest.texというファイルを作成したとします。これをコンパイルするには、ptex2pdfというコマンドを以下のように使用します。

\centerline{\$\ ptex2pdf -l -ot -kanji=utf-8 -synctex=1 test.tex}
作成したpdfファイルの出力先を指定するには、"-output-directory (ディレクトリ名)"を指定します。

\section{Basic Usage}

まずは数式ではなく、基本的な文章を作るための最初のフォーマットを紹介します。

\begin{lstlisting}[caption=LaTeX]
% '%' から文字列を始めることによって、LaTeX 文章内にコメントを書くことができる

\documentclass[paper=b5]{jlreq}
% documentclass にjlreq を指定、出力の用紙サイズをB5 に
\usepackage{bxpapersize}
% pdf の用紙サイズを変更するためのパッケージ

\begin{document}
% ここから文章を書いていく

\title{Hello, \LaTeX !}
% 文章タイトルの設定
\author{Standard}
% 著者の設定
\maketitle
% これらの設定でタイトルを表示

\section{First \LaTeX document}
% セクションの表示、この文章で言うところの"Basic Usage" とか

Hello!
% そのまま書くと普通に表示される

\end{document}
% ここまでが文章

\end{lstlisting}

まず、3行目でdocumentclassの指定を行います。これは\LaTeX の文章を作成する上での設定ファイルのようなもので、このdocumentclassに含まれる情報と、文章のファイルに書かれた情報でpdfを作成します。
このdocumentclassにはオプションをつけることができ、この例ではpaper=b5を指定してpdfの文章のサイズを指定しています。

5行目はpdfの紙サイズを指定するためのパッケージを使うことを宣言しています。パッケージとは、LaTeXにコマンドや機能を追加するための物で、いろいろな用途のための物が存在します。このパッケージでは、3行目でpaper=b5を指定しましたが、このままではpdfのA4のサイズにB5の文章を印刷したような状態になってしまいます、そこでその紙サイズも文章サイズに合わせるためにパッケージを持ってきています。\footnote{よく使うパッケージはさっきのインストールで導入済みのため、新たにインストールする必要はないです。}

8行目の\verb|\begin{document}|から24行目の\verb|\end{document}|までをこれらの宣言で挟むことで、その間にかいたものが最終的な文章に含まれる部分となります。HTMLで言うところの\verb|<body><\body>|に当たるものですね。これらの\verb|\begin|で始め、\verb|\end|で閉じることで作成される部分は環境と呼ばれ、document環境以外にも図を表示するためのfigure環境や数式を記述するためのequation環境、表を作るためのtable環境などが存在します。

11行目からはタイトルの設定です。タイトル内でもLaTeXの構文は有効なので今回のようにタイトル内で\LaTeX のような特殊な文字を入れることも可能です。

18行目、\verb|\section|は段落を表示する宣言です。

そして21行目のように何も宣言せずにベタ書きをすると、本文の文章となります。

\section{Writing Formula}

\LaTeX で数式を書くのは簡単です。結局数式なんていくつかの記号の組み合わせなのでパパっと覚えちゃいましょう。

\begin{table}[H]
  \centering
  \begin{tabular}{l|l}
    $\verb|p \div q \times r = \frac{pr}{q}|$                               &
    $\displaystyle p \div q \times r = \frac{pr}{q}$                          \\

    $\verb|\left\lfloor x\right\rfloor, \left\lceil y\right\rceil|$         &
    $\displaystyle \left\lfloor x\right\rfloor, \left\lceil y\right\rceil$    \\

    $\verb|\sqrt{b^2 - 4ac}|$                                               &
    $\displaystyle \sqrt{b^2 - 4ac}$                                          \\

    $\verb|\sqrt[3]{27} = 3|$                                               &
    $\displaystyle \sqrt[3]{27} = 3$                                          \\

    $\verb|7 \equiv 1 \pmod 3|$                                             &
    $\displaystyle 7 \equiv 1 \pmod 3$                                        \\

    $\verb|a \leqq b| \tablefootnote{amssymbパッケージの使用が必要です。\label{amssymb}}$ &
    $\displaystyle a \leqq b$                                                 \\

    $\verb|c \geqq d| \textsuperscript{\getrefnumber{amssymb}}$             &
    $\displaystyle c \geqq d$                                                 \\

    $\verb|e \pm f \mp g|$                                                  &
    $\displaystyle e \pm f \mp g$                                             \\

    $\verb|\sin \theta|$                                                    &
    $\displaystyle \sin \theta$                                               \\

    $\verb|\cos \phi|$                                                      &
    $\displaystyle \cos \phi$                                                 \\

    $\verb|\tan \angle BAC|$                                                &
    $\displaystyle \tan \angle BAC$                                           \\

    $\verb|\log_{10} n|$                                                    &
    $\displaystyle \log_{10} n$                                               \\

    $\verb|\overrightarrow{AB} = \vec{b} - \vec{a}|$                        &
    $\displaystyle \overrightarrow{AB} = \vec{b} - \vec{a}$                   \\

    $\verb"|\vec{b}|^2 - 2\vec{b} \cdot \vec{a} + |\vec{a}|^2"$             &
    $\displaystyle |\vec{b}|^2 - 2\vec{b} \cdot \vec{a} + |\vec{a}|^2$        \\

    $\verb"\|\vec{b}\|^2 - 2\vec{b} \cdot \vec{a} + \|\vec{a}\|^2"$         &
    $\displaystyle \|\vec{b}\|^2 - 2\vec{b} \cdot \vec{a} + \|\vec{a}\|^2$    \\

    $\verb|{}_{n}\mathrm{C}_r|$                                             &
    ${}_{n}\mathrm{C}_r$                                                      \\

    % $\verb|\binom{r}{n}| \tablefootnote{ベクトルの成分表示ではなく二項係数です}$               &
    % $\displaystyle \binom{n}{r}$                                              \\

    $\verb|n \in \mathbb{N}|$                                               &
    $\displaystyle n \in \mathbb{N}$                                          \\

    $\verb|A \Rightarrow B \Leftrightarrow C|$                              &
    $\displaystyle A \Rightarrow B \Leftrightarrow C$                         \\

    $\verb|1 + \frac{1}{4} + \frac{1}{9} + \cdots|$                         &
    $\displaystyle 1 + \frac{1}{4} + \frac{1}{9} + \cdots$                    \\

    $\verb|\sum_{k=1}^{n}|$                                                 &
    $\displaystyle \sum_{k=1}^{n} k^2$                                        \\

    $\verb|\lim_{n \to \infty}|$                                            &
    $\displaystyle \lim_{n \to \infty}$                                       \\

    $\verb|\int_{a}^{b}|$                                                   &
    $\displaystyle \int_{a}^{b}$                                              \\
  \end{tabular}
\end{table}

簡単な規則性としては、アンダーバー(\_)を付けると下に付き、キャレット(\verb|^|)を付けると上に付きます。これはコマンドでなくとも機能し、a\_nと書いてやると$a_n$と表示することができます。複数の文字をこのように付けるには\{ \}で囲む必要があります\verb|a_{under}^{upper}|と書けば$a_{under}^{upper}$このように表示されます。

数式は地の文として記述しても表示できません。数式を書くには数式たのめの環境が必要です。
数式を書くための環境は2種類あります。まずは文中に書くためのテキストスタイルです。
テキストスタイルでの記述環境は\verb|$|で挟むことで作成できます。例えば\verb|$ax^2+bx+c=0 \quad (a \neq 0)$|\footnote{$\backslash$quadは空白を空けるためのコマンド}と記述すれば、$ax^2+bx+c=0 \quad (a \neq 0)$と表示できます。
次に、ディスプレイスタイルです。これは、文中ではなく普通に数式を段落を変えて書くときの記法で、テキストスタイルと違い、\verb|$$|と、ドルマーク2つで挟むことでできます。
例えば\verb|$$I_n = \int_{0}^{1} x^n e^x dx$$|と記述してやると、
$$
  I_n = \int_{0}^{1} x^n e^x dx
$$
というように表示できます。

他にも環境を作ってその中に数式を記述する方法があり、こちらの方が書くのは少し手間ですが自由度は高いです。環境はいくつかあるのですが、まず始めにequation環境です。これは一行の数式を番号付きで表示するためのものです。

\begin{lstlisting}[caption=LaTeX]
\begin{equation}
  x^n + y^n = z^n \qquad (n \geqq 3)
\end{equation}
\end{lstlisting}

\begin{equation}
  x^n + y^n = z^n \qquad (n \geqq 3)
\end{equation}

gather環境、これは複数行の数式を中央揃えで番号を付けて表示します。改行したい場所で\verb|\\|を書きましょう。
例えばこのように書けば

\begin{lstlisting}[caption=LaTeX]
\begin{gather}
  \int_{a}^{b} kf (x) dx = k \int_{a}^{b} f(x) dx
  \int_{a}^{b} {f(x) + g(x)} dx = \int_{a}^{b} f(x) dx + \int_{a}^{b} g(x) dx
  \int_{a}^{b} f(x) dx = \int_{a}^{b} f(t) dt
\end{gather}
\end{lstlisting}

\begin{gather}
  \int_{a}^{b} kf (x) dx = k \int_{a}^{b} f(x) dx \\
  \int_{a}^{b} {f(x) + g(x)} dx = \int_{a}^{b} f(x) dx + \int_{a}^{b} g(x) dx \\
  \int_{a}^{b} f(x) dx = \int_{a}^{b} f(t) dt \\
\end{gather}

数式番号を付けたくない行には\verb|\notag|コマンドを使います。また、*を付けて\verb|\begin{gather*}|や\verb|\begin{equation*}|のように書けば、全ての行に対して番号を付けないことが可能です。逆に*を付けたとしても、\verb|\tag{hoge}|のように書くことで任意のタグを付けることも可能です。\verb|\tag*{fuga}|とすることで括弧で囲わないことも可能です。

数式を=で揃えたい場合にはalign環境を使います。揃える場所で\verb|&|を書くことでその位置で揃います。

\begin{lstlisting}[caption=LaTeX]
\begin{align}
  \cos 2\theta &= \cos^2 \theta - \sin^2 \theta \tag{abcd} \\
  &= 2\cos^2 \theta + 1 \\
  &= 1 - 2\sin^2 \theta
\end{align}
\end{lstlisting}

\begin{align}
  \cos 2\theta & = \cos^2 \theta - \sin^2 \theta \tag{important} \\
               & = 2\cos^2 \theta + 1                            \\
               & = 1 - 2\sin^2 \theta
\end{align}

align環境は複数の\verb|&|で揃えることもできます。このとき、奇数番目の\verb|&|は揃える位置を、偶数番目の\verb|&|は区切りを表します。この区切りがなければ、どこまでを左の\verb|&|に合わせて、どこまでを右に合わせるのかが分からないからです。なおこの時、区切りが入った場所にはスペースが入ります。

\begin{lstlisting}[caption=LaTeX]
\begin{align*}
  a_1 & = c_1 & a_2+b_2 & = c_2     \\
  a_3 & = c_3 & a_4     & = c_4+d_4
\end{align*}
\end{lstlisting}

\begin{align*}
  a_1 & = c_1 & a_2+b_2 & = c_2     \\
  a_3 & = c_3 & a_4     & = c_4+d_4
\end{align*}

このままでは不便なのでスペースを入れたくない場合はalignat環境を使います。この環境は(各行の\verb|&|の個数の最大値 + 1)/2を指定することで、区切り文字の場所にスペースが入らなくなります。

\begin{lstlisting}[caption=LaTeX]
\begin{alignat}{2}
  10x & {}+{} & 2y  & = 0 \\
  5x  & {}+{} & 37y & = 9
\end{alignat}
\end{lstlisting}

\begin{alignat}{2}
  10x & {}+{} & 2y  & = 0 \\
  5x  & {}+{} & 37y & = 9
\end{alignat}

ここで、\verb|{}+{}|としているのは、こうしないと+が二項演算子の意味にならず、文字の間隔がおかしくなるためです。

環境中に環境を組み込むこともできます。align環境と同様の機能を持つaligned環境やalignat版のalignedatを使うことで、これらの環境を環境の中で使うことができます。複数行の数式を括弧で囲いたい時などに使います。

\begin{lstlisting}[caption=LaTeX]
\begin{equation*}
  f(n) = \left\{
  \begin{alignedat}{2}
     & n / 2  & \quad\mathrm{if} n & \equiv 0 \\
     & 3n + 1 & \quad\mathrm{if} n & \equiv 1
  \end{alignedat}
  \right. \pmod 2
\end{equation*}
\end{lstlisting}

\begin{equation*}
  f(n) = \left\{
  \begin{alignedat}{2}
     & n / 2  & \quad\mathrm{if} n & \equiv 0 \\
     & 3n + 1 & \quad\mathrm{if} n & \equiv 1
  \end{alignedat}
  \right. \pmod 2
\end{equation*}

\section{Vector and Matrix}

行列の書き方はこうします。雰囲気としてはalignと同じようなものだと思ってもらえれば大丈夫です。

\begin{lstlisting}[caption=LaTeX]
\begin{equation*}
  \begin{matrix}
    a & b \\
    c & d \\
  \end{matrix}
\end{equation*}
\end{lstlisting}

\begin{equation*}
  \begin{matrix}
    a & b \\
    c & d \\
  \end{matrix}
\end{equation*}

他の書き方での行列及び行列式は上のmatrix部分を以下の文字列に置き換えるだけです。


\begin{table}[H]
  \centering
  \begin{tabular}{c|c||c|c}
    pmatrix                        &
    $\displaystyle \begin{pmatrix}
                       a & b \\
                       c & d
                     \end{pmatrix}$ &
    bmatrix                        &
    $\displaystyle \begin{bmatrix}
                       a & b \\
                       c & d
                     \end{bmatrix}$    \\
    vmatrix                        &
    $\displaystyle \begin{vmatrix}
                       a & b \\
                       c & d
                     \end{vmatrix}$ &
    Vmatrix                        &
    $\displaystyle \begin{Vmatrix}
                       a & b \\
                       c & d
                     \end{Vmatrix}$    \\
  \end{tabular}
\end{table}

\verb|\cdots| $\cdots$ や\verb|\vdots| $\vdots$ 、\verb|\ddots| $\ddots$ を駆使することでこのような行列を作ることができます。

\begin{lstlisting}[caption=LaTeX]
\begin{equation*}
  \begin{pmatrix}
    1      & 0      & \cdots & 0      \\
    0      & 1      & \cdots & 0      \\
    \vdots & \vdots & \ddots & \vdots \\
    0      & 0      & \cdots & 1
  \end{pmatrix}
\end{equation*}
\end{lstlisting}

\begin{equation*}
  \begin{pmatrix}
    1      & 0      & \cdots & 0      \\
    0      & 1      & \cdots & 0      \\
    \vdots & \vdots & \ddots & \vdots \\
    0      & 0      & \cdots & 1
  \end{pmatrix}
\end{equation*}

ベクトルの成分表示は$1 \times n$行列として考えれば簡単です。

\begin{lstlisting}[caption=LaTeX]
\begin{gather*}
  \vec{a} = \begin{pmatrix}
    x_a \\ y_a
  \end{pmatrix},
  \vec{b} = \begin{pmatrix}
    x_b \\ y_b
  \end{pmatrix} \\
  \vec{a} \cdot \vec{b} = x_a x_b + y_a y_b
\end{gather*}
\end{lstlisting}

\begin{gather*}
  \vec{a} = \begin{pmatrix}
    x_a \\ y_a
  \end{pmatrix},
  \vec{b} = \begin{pmatrix}
    x_b \\ y_b
  \end{pmatrix} \\
  \vec{a} \cdot \vec{b} = x_a x_b + y_a y_b
\end{gather*}

\section{Insert Chart and Figure}

文章中に表を挿入するにはtable環境とtabular環境を使います。

\begin{lstlisting}[caption=LaTeX]
\begin{table}[htbp]
  \centering
  \caption{プログラミング言語一覧}
  \begin{tabular}{l|ll}
    言語       & 年代 & 開発                     \\ \hline
    Fortran    & 1957 & John Warner Backus       \\
    COBOL      & 1959 & Grace Murray Hopper      \\
    (中略)
    Rust       & 2010 & Graydon Hoare            \\
  \end{tabular}
\end{table}
\end{lstlisting}

\begin{table}[htbp]
  \centering
  \caption{プログラミング言語一覧}
  \begin{tabular}{l|ll}
    言語         & 年代   & 開発                       \\ \hline
    Fortran    & 1957 & John Warner Backus       \\
    COBOL      & 1959 & Grace Murray Hopper      \\
    LISP       & 1960 & John McCarthy            \\
    PL/I       & 1964 & IBM                      \\
    BASIC      & 1964 & Dartmouth College        \\
    C          & 1972 & Bell Laboratories        \\
    C++        & 1983 & Bjarne Stroustrup        \\
    Python     & 1991 & Guido van Rossum         \\
    Java       & 1995 & Sun Microsystems(Oracle) \\
    JavaScript & 1995 & Brendan Eich             \\
    C\#        & 2000 & Microsoft                \\
    Rust       & 2010 & Graydon Hoare            \\
  \end{tabular}
\end{table}

まず始めにtable環境を作成し、その中でtabular環境を宣言します。
表を文章の中心に揃えたければ\verb|\centering|宣言をします。\verb|\caption{キャプション}|を使うことで表の上部にキャプションを付けることもできます。
tabluar環境には引数を与えます列の数だけその列の中心に要素を置きたければc、左側に置きたければl、右側ならrです。それぞれの文字の間に|を置くことでその本数だけ縦線を引くことも可能です。\verb|\hline|ではその列に横線を引くことができます、こちらも書けば書くだけ本数は増えます。

\section{Physics}

\section{Chemistry}

\section{Insert References}

\section{Other Information}

\section{Postscript}

夏休み前に書き始めたのに思ったより文量が多くなったのと、だらだらと書いていたのでたちばな祭に間に合いませんでした。受験生の身ではありますが、これは最後まで書き上げたいと思うので完全版は予餞会までお待ちください。作業用のリポジトリを公開するので、早く見たいという方はこのリンクをどうぞ(https://github.com/Standard8989/World-of-LaTeX)。

\end{document}
